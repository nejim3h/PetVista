\documentclass{article}
\usepackage[utf8]{inputenc}
\usepackage{amsmath}
\usepackage{algorithm}
\usepackage{algpseudocode}
\usepackage{xcolor}

\usepackage[margin=2.2cm]{geometry}

\title{Project Deadline-5: DBMS Winter-2024} \author{Nimit Panwar (2022324)}

\date{31 March 2024} \begin{document}

\maketitle

\vspace{3mm}

\section*{Ordering items and Inventory Analysis}

\vspace{3mm}

\begin{enumerate}
    \item \textbf{Get Product Query:}
    \begin{verbatim}
$get_product = "select * from products where product_url='\$product_id'";
    \end{verbatim}
    This query is used to select all fields from the \texttt{products} table where the \texttt{product\_url} field matches the \texttt{\$product\_id} variable.
    
    \item \textbf{Get Product Category Query:}
    \begin{verbatim}
$get_p_cat = "select * from product_categories where p_cat_id='\$p_cat_id'";
    \end{verbatim}
    This query is used to select all fields from the \texttt{product\_categories} table where the \texttt{p\_cat\_id} field matches the \texttt{\$p\_cat\_id} variable.
    
    \item \textbf{Check Product Query:}
    \begin{verbatim}
$check_product = "select * from cart where ip_add='\$ip_add' AND p_id='\$p_id'";
    \end{verbatim}
    This query is used to select all fields from the \texttt{cart} table where the \texttt{ip\_add} field matches the \texttt{\$ip\_add} variable and the \texttt{p\_id} field matches the \texttt{\$p\_id} variable.
    
    \item \textbf{Get Price Query:}
    \begin{verbatim}
$get_price = "select * from products where product_id='\$p_id'";
    \end{verbatim}
    This query is used to select all fields from the \texttt{products} table where the \texttt{product\_id} field matches the \texttt{\$p\_id} variable.
    
    \item \textbf{Insert into Cart Query:}
    \begin{verbatim}
$query = "insert into cart (p_id,ip_add,qty,p_price,size) values 
          ('\${p_id}','\${ip_add}','\${product_qty}',
          '\${product_price}','\${product_size}')"; 
    \end{verbatim}
    This query is used to insert a new row into the \texttt{cart} table with the specified values.
    
    \item \textbf{Get Customer Query:}
    \begin{verbatim}
$get_customer = "select * from customers where customer_email='\$customer_session'";
    \end{verbatim}
    This query is used to select all fields from the \texttt{customers} table where the \texttt{customer\_email} field matches the \texttt{\$customer\_session} variable.
    
    \item \textbf{Select Wishlist Query:}
    \begin{verbatim}
$select_wishlist = "select * from wishlist where customer_id='\$customer_id' 
                   AND product_id='\$pro_id'";
    \end{verbatim}
    This query is used to select all fields from the \texttt{wishlist} table where the \texttt{customer\_id} field matches the \texttt{\$customer\_id} variable and the \texttt{product\_id} field matches the \texttt{\$pro\_id} variable.
    
    \item \textbf{Insert into Wishlist Query:}
    \begin{verbatim}
$insert_wishlist = "insert into wishlist (customer_id,product_id) 
                   values ('\${customer_id}','\${pro_id}')"; 
    \end{verbatim}
    This query is used to insert a new row into the \texttt{wishlist} table with the specified values.
    
    \item \textbf{Get Products Query:}
    \begin{verbatim}
$get_products = "select * from products order by rand() LIMIT 0,3";
    \end{verbatim}
    This query is used to select all fields from the \texttt{products} table in a random order, limiting the results to the first 3.
    
    
    \item \textbf{Get Email for Login Query:}
    \begin{verbatim}
$get_email = "select * from customers where customer_email='$c_email'";
    \end{verbatim}
    This query is used to select all fields from the \texttt{customers} table where the \texttt{customer\_email} field matches the \texttt{\$c\_email} variable. It's used to check if the email is already registered.
    
    \item \textbf{Register Query:}
    \begin{verbatim}
$insert_customer = "insert into customers                                       
                    (customer_name,customer_email,customer_pass,customer_country,
                    customer_city,customer_contact,customer_address,
                    customer_image,customer_ip,customer_confirm_code) values 
                    ('$c_name','$c_email','$c_pass','$c_country','$c_city',
                    '$c_contact','$c_address','$c_image','$c_ip',
                    '$customer_confirm_code')";
    \end{verbatim}
    This query is used to insert a new row into the \texttt{customers} table with the specified values. It's used to register a new customer.

    \item \textbf{Select Cart Query:}
    \begin{verbatim}
$select_cart = "select * from cart where ip_add='$ip_add'";
    \end{verbatim}
    This query is used to select all fields from the \texttt{cart} table where the \texttt{ip\_add} field matches the \texttt{\$ip\_add} variable. It's used to fetch all the items in the cart for the current user.
    
    \item \textbf{Get Products Query:}
    \begin{verbatim}
$get_products = "select * from products where product_id='$pro_id'";
    \end{verbatim}
    This query is used to select all fields from the \texttt{products} table where the \texttt{product\_id} field matches the \texttt{\$pro\_id} variable. It's used to fetch the details of a specific product.
    
    \item \textbf{Get Cart Query:}
    \begin{verbatim}
$get_cart = "select * from cart where p_id='$coupon_pro' AND ip_add='$ip_add'";
    \end{verbatim}
    This query is used to select all fields from the \texttt{cart} table where the \texttt{p\_id} field matches the \texttt{\$coupon\_pro} variable, and the \texttt{ip\_add} field matches the \texttt{\$ip\_add} variable. It's used to check if the product associated with the coupon is in the cart.
    
    \item \textbf{Update Cart Query:}
    \begin{verbatim}
$update_cart = "update cart set p_price='$coupon_price' where p_id='$coupon_pro' 
                AND ip_add='$ip_add'";
    \end{verbatim}
    This query is used to update the \texttt{p\_price} field in the \texttt{cart} table where the \texttt{p\_id} field matches the \texttt{\$coupon\_pro} variable, and the \texttt{ip\_add} field matches the \texttt{\$ip\_add} variable. It's used to apply the coupon discount to the product in the cart.
    
    \item \textbf{Delete Product from Cart Query:}
    \begin{verbatim}
$delete_product = "delete from cart where p_id='$remove_id'";
    \end{verbatim}
    This query is used to delete a row from the \texttt{cart} table where the \texttt{p\_id} field matches the \texttt{\$remove\_id} variable. It's used to remove a product from the cart.
    
    \item \textbf{Insert Customer Order Query:}
    \begin{verbatim}
$insert_customer_order = "insert into customer_orders (customer_id,due_amount,
invoice_no,qty,size,order_date,order_status) values ('$customer_id','$sub_total',
'$invoice_no','$pro_qty','$pro_size',NOW(),'$status')";
    \end{verbatim}
    This query is used to insert a new row into the \texttt{customer\_orders} table with the specified values. It's used to create a new order for the customer.
    
    \item \textbf{Insert Pending Order Query:}
    \begin{verbatim}
$insert_pending_order = "insert into pending_orders (customer_id,invoice_no,
product_id,qty,size,order_status) values ('$customer_id','$invoice_no',
'$pro_id','$pro_qty','$pro_size','$status')";
    \end{verbatim}
    This query is used to insert a new row into the \texttt{pending\_orders} table with the specified values. It's used to add the order to the list of pending orders.
    
    \item \textbf{Delete Cart Query:}
    \begin{verbatim}
$delete_cart = "delete from cart where ip_add='$ip_add'";
    \end{verbatim}
    This query is used to delete all rows from the \texttt{cart} table where the \texttt{ip\_add} field matches the \texttt{\$ip\_add} variable. It's used to clear the cart after the order has been placed.
    
\end{enumerate}


\section*{Customer Analysis}

\vspace{3mm}

\begin{enumerate}
    \item \textbf{Get Product Categories Query:}
    \begin{verbatim}
$get_p_categories = "SELECT * FROM product_categories";
    \end{verbatim}
    This query selects all fields from the \texttt{product\_categories} table. It's used to fetch all the product categories.
    
    \item \textbf{Get Total Orders Query:}
    \begin{verbatim}
$get_total_orders = "SELECT * FROM customer_orders";
    \end{verbatim}
    This query selects all fields from the \texttt{customer\_orders} table. It's used to fetch all the customer orders.
    
    \item \textbf{Get Pending Orders Query:}
    \begin{verbatim}
$get_pending_orders = "SELECT * FROM customer_orders WHERE order_status='pending'";
    \end{verbatim}
    This query selects all fields from the \texttt{customer\_orders} table where the \texttt{order\_status} field is 'pending.' It's used to fetch all the pending customer orders.
    
    \item \textbf{Get Completed Orders Query:}
    \begin{verbatim}
$get_completed_orders = "SELECT * FROM customer_orders WHERE order_status='Complete'";
    \end{verbatim}
    This query selects all fields from the \texttt{customer\_orders} table where the \texttt{order\_status} field is 'Complete.' It's used to fetch all the completed customer orders.
    
    \item \textbf{Get Total Earnings Query:}
    \begin{verbatim}
$get_total_earnings = "SELECT SUM(due_amount)
                    as Total FROM customer_orders WHERE order_status = 'Complete'";
    \end{verbatim}
    This query calculates the sum of the \texttt{due\_amount} field from the \texttt{customer\_orders} table where the \texttt{order\_status} field is 'Complete.' It's used to calculate the total earnings from completed orders.
    
    \item \textbf{Get Coupons Query:}
    \begin{verbatim}
$get_coupons = "SELECT * FROM coupons";
    \end{verbatim}
    This query selects all fields from the \texttt{coupons} table. It's used to fetch all the coupons.
\end{enumerate}


\vspace{5mm}


\section*{Triggers}

\vspace{3mm}

1. Automatically increments coupon usage count in \texttt{coupons} table upon usage and insertion into \texttt{cart}.

\begin{verbatim}
        DELIMITER //
        CREATE TRIGGER update_coupon_used
        AFTER INSERT ON cart
        FOR EACH ROW
        BEGIN
            DECLARE coupon_limit INT;
            DECLARE coupon_used INT;
        
            SELECT coupon_limit, coupon_used INTO coupon_limit, coupon_used
            FROM coupons
            WHERE coupon_id = NEW.product_id;
        
            IF coupon_used < coupon_limit THEN
                UPDATE coupons
                SET coupon_used = coupon_used + 1
                WHERE coupon_id = NEW.product_id;
            END IF;
        END;
        //
        DELIMITER ;
\end{verbatim}

2. Updates pending order status to Complete upon payment insertion

\begin{verbatim}
        DELIMITER //
        CREATE TRIGGER update_order_status_on_payment
        AFTER INSERT ON payment
        FOR EACH ROW
        BEGIN
            UPDATE pending_orders
            SET order_status = 'Complete'
            WHERE invoice_no = NEW.invoice_no;
        END;
        //
        DELIMITER ;

\end{verbatim}

\end{document}
